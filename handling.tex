\chapter{Data handling}
\label{chap:handling}

\chapterprecishere{%
  Tidy datasets are all alike, but every messy dataset is messy in its own way.
  \par\raggedleft--- \textup{Hadley Wickham}, Tidy Data}

% Important: avoid the term "data manipulation" as it has a negative connotation

Data handling is the process of adjusting data to make it suitable for analysis.
It involves three main tasks: data transformation, data cleaning, and data integration.

\section{Data handling operators}

In the literature and in software documentation, you will find a variety of terms used to
describe data handling operations\footnote{%
  The terminology ``data handling'' itself is not universal.  Some authors and libraries
  call it ``data manipulation'', ``data wrangling'', ``data shaping'', or ``data
  engineering''.  I use the term ``data handling'' to avoid confusion with the term ``data
  manipulation'' which has a negative connotation in some contexts.}. %
They often refer to the same or similar operations, but the terminology can be confusing.
In this section, I present a summary of these operations mostly based on
\textcite{Wickham2023} definitions\footnote{Which are called \emph{verbs}.}.

\begin{slidebox}{Data handling operators}{}
  \begin{itemize}
    \item Filtering rows;
    \item Selecting columns;
    \item Mutating columns;
    \item Aggregating rows;
    \item Binding datasets;
    \item Joining datasets;
    \item Pivoting (spreading) and unpivoting (gathering) datasets.
  \end{itemize}
\end{slidebox}

These operations are the building blocks of the data handling tasks we will discuss in the
next sections.  They can also be extensively parametrized and combined to create more
elaborate data handling pipelines.  For instance, most of them can use predicates to
define the groups, arrangements, or conditions under which they should be applied.

We use the following terminology to refer to the data handling parameters:
\begin{itemize}
  \item \textbf{Predicate}: a function that returns a logical value, used to filter
    rows/columns or to define the groups of rows/columns to be processed;
  \item \textbf{Aggregation function}: a function that returns a single value given a vector
    of values (in which, the order of the values may be important);
  \item \textbf{Window function}: a function that returns a vector of values given a vector
    of values in which, the order of the values is important;
  \item \textbf{Expression}: a function that returns a vector of values, used to create new
    columns or to modify existing ones.
\end{itemize}

\begin{slidebox}{Data handling pipelines}{}
  \begin{itemize}
    \item Data handling operations can be combined to create complex pipelines;
    \item Operators may be reversible;
    \item Operators are vectorized;
    \item They can be parametrized with predicates, aggregation functions, and expressions;
    \item They operate on datasets and return new datasets as output.
    \item They are declarative.
  \end{itemize}
\end{slidebox}

Operators are also vectorized, meaning that they can be applied to multiple columns or
rows at once.  This is a key feature of data handling operations, as it allows for
expressive and efficient data manipulation.

Many of them are also reversible, meaning that they can be undone.  This is important
because it allows for reproducibility and traceability of the data handling process.

They operate on a dataset (or more than one) given as input and return a new dataset as
output.  This is important because it allows for the creation of data handling pipelines,
where the output of one operation is the input of the next one.  Parameters like column
names, predicates, aggregation functions, and expressions can be passed to these operations to
customize their behavior.

Unlike traditional procedural programming, where conditional statements and loops are used
to manipulate data, data handling operations are declarative.  This means that they are
expressed in terms of what should be done, not how it should be done.  This is a powerful
abstraction that allows for the creation of complex pipelines with a few lines of code.

\subsection{Filtering rows}

Filtering is the process of selecting a subset of rows from a dataset based on a
predicate.  If more than a single predicate is used, they are combined using a logical
operator, such as \texttt{AND} or \texttt{OR}.

After filtering, the dataset will contain only the rows that satisfy the predicate.
Columns remain unchanged.  This operation is potentially irreversible, as the removed
rows are lost.

In the basic form, each row is treated independently.  For instance, the predicate
\texttt{age > 18} will select all rows where the value in the \texttt{age} column is
greater than 18.

However, if the predicate depends on a aggregation or window function, one must specify
the groups and/or the order of the rows.  For instance, the predicate \texttt{age >
mean(age) group by country} will select the rows where the value in the \texttt{age}
column is greater than the mean of the \texttt{age} for each \texttt{country}. Another
example is the predicate \texttt{cumsum(price) < 100 sort by date}, which selects the rows
that satisfy the condition that the cumulative sum of the \texttt{price} column is less
than 100 given the order of the rows defined by the \texttt{date} column.

The trivial group is the entire dataset, so it is usually not necessary to specify it
explicitly.  However, it is usually not sensible to not specify the order of the rows.

When dealing with real values, be aware of floating-point precision issues.  In other
words, do not use the equality operator to compare real numbers.  Most of libraries
provide operators to compare real numbers within a given tolerance.

\begin{mainbox}{Practical tips}
  \begin{itemize}
    \item Use filtering to remove rows that are not relevant to your analysis;
    \item Use predicates to define the conditions under which rows should be removed;
    \item When aggregation functions are needed to define the predicate, specify the groups and
      the order of the rows;
    \item Be aware of floating-point precision issues when comparing real numbers.
  \end{itemize}
\end{mainbox}

\subsection{Selecting columns}

Selecting is the process of choosing a subset of columns from a dataset.  The remaining
columns are discarded.  This operation is not reversible, as the discarded columns are
lost.  Rows remain unchanged.

There are two main ways to select columns: by name or by predicate.  The former is the
most common and is used to select a fixed set of columns.  The latter is used to select
columns that satisfy a given condition, i.e., the values in the columns are used to
determine which columns should be selected.

When selecting columns by name, one can use a list of column names or a regular
expression.  The latter is useful when the column names follow a pattern.  For instance,
one can use the regular expression \texttt{col[0-9]+} to select all columns whose names
start with \texttt{col} followed by one or more digits.

When selecting columns by predicate, one can use a function that returns a logical value
to define the condition under which a column should be selected.  For instance, one can
use the predicate \texttt{isnumeric} to select all columns that contain numeric values.
Notice, however, that the predicate is applied to each column independently and returns a
single logical value for each column.

Like filtering, selecting predicates might contain aggregation functions.  Although it is
theorically possible to consider the order of the values in the columns, it is not common
to do so.  (Especially because one would need to assume that the rows are previously
sorted by some criterion.) Groups, however, never make sense in this context, once the
predicate is applied to each column independently.

Depending on the context, it may be useful to ``drop'' columns instead of selecting them.
This is the same as selecting all columns except the ones specified.  This is useful when
the number of columns to be dropped is small compared to the total number of columns.
Strictly speaking, we just need to negate the predicate or the regular expression used to
select the columns.

Finally, it is very common to find libraries and framework in which the order of the
columns is important.  As a result, columns can be selected by position as well.
I find this practice error-prone and I recommend avoiding it whenever possible.

\begin{mainbox}{Practical tips}
  \begin{itemize}
    \item Use selecting to remove columns that are not relevant to your analysis;
    \item Use column names or regular expressions to select columns;
    \item Use predicates (many to one, with no aggregation functions) to define the conditions
      under which columns should be selected;
    \item Avoid depending on the order of the columns.
  \end{itemize}
\end{mainbox}

\subsection{Mutating columns}

Mutating is the process of creating new columns.  The operation is reversible, as the
original columns are kept.  The new columns are added to the dataset.

The values in the new column are determined by an expression.  The expression is a
function that returns a vector of values given the values in the other columns.  The
expression can be a simple function, such as \texttt{y = x + 1}, or a more complex
function, such as \texttt{y = ifelse(x > 0, 1, 0)}.  Here, \texttt{x} and \texttt{y} are
the names of an existing and the new column, respectively.

One may also use a aggregation and window function in the expression. This is particularly
useful when performing mutation considering a group.  In this case, the returned value is
repeated (aggregation function) for each row of the same group.  Like in filtering, the
more explicit you can be about order and groups, the better.

For example, the expression \texttt{y = cumsum(x) group by category sort by date} will
create a new column \texttt{y} with the cumulative sum of the \texttt{x} column for each
\texttt{category} given the order of the rows defined by the \texttt{date} column.

Sometimes, the same expression can be used to create multiple columns.  This is useful
when the new columns are related.  To do so, one first specifies the columns in the same way as
when selecting columns.  Then, one needs to specify a rule to name the new columns.
For instance, \texttt{x\_new = x + 1 across x matches \textasciicircum{}col[0-9]+\$}.

Practically speaking, mutation can overwrite existing columns.  This is useful when the
new column is a replacement for the old one.  Formally, overwriting is just a sequence of
mutation and selection operations.

\begin{mainbox}{Practical tips}
  \begin{itemize}
    \item Use mutating to create new columns that are relevant to your analysis;
    \item Use expressions to define the values of the new columns;
    \item Use aggregation and window functions in the expression to create new columns based on
      groups and order;
    \item Use the same expression to create multiple columns when the new columns are related.
  \end{itemize}
\end{mainbox}

\subsection{Aggregating rows}

Aggregating is the process of reducing the number of rows in a dataset.  The operation is
not reversible, as the discarded rows are lost.  The columns are also lost, only the new
aggregate columns remain.

The values in the new columns are determined by an aggregation function.  Like filtering
and mutation, the aggregation function can be parametrized by specifying a group and/or an
order.

The resulting dataset will contain one row for each group.  The values in the new columns
are determined by the aggregation function applied to the values in the other columns.

\section{Data transformation}

\subsection{Reshaping}

\subsection{Sampling}

\subsection{Dimensionality reduction}

\subsection{Normalization}

\begin{mainbox}{Practice!}
  Can you identify which data transformation operations are used to make datasets
  presented in \cref{chap:data} tidy?
\end{mainbox}

\section{Data cleaning}

\subsection{Missing data}

\subsection{Outliers}

\subsection{Noise}

\section{Data integration}

\subsection{Joining}

\subsection{Merging}

\subsection{Concatenating}
