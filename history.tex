\chapter{Brief history of data science}

There are many points-of-view about the beginning of data science.  For the sake of
contextualization, I separate the topic in three approaches: the history of the term,
important figures and other historical markers, and a broad timeline of data-driven
sciences.

\section{The term ``data science''}

\paragraph{Peter Naur (1928 -- 2016)}

The term ``data science'' itself was coined in the 1960s by Peter Naur (/naʊə/). Naur was
a Danish computer scientist and mathematician who made significant contributions to the
field of computer science, including his work on the development of programming
languages\footnote{He is best remembered as a contributor, with John Backus, to the
Backus–Naur form (BNF) notation used in describing the syntax for most programming
languages.}.
His ideas and concepts laid the groundwork for the way we think about programming and data
processing today.

Naur disliked the term computer science and suggested it be called datalogy or data
science.  In the 1960s, the subject was practised in Denmark under Peter
Naur's term datalogy, which means the science of data and data processes.

He coined this term to emphasize the importance of data as a fundamental component of
computer science and to encourage a broader perspective on the field that included
data-related aspects. At that time, the field was primarily centered on programming
techniques, but Naur's concept broadened the scope to recognize the intrinsic role of data
in computation.

In his book\footnote{Peter Naur: Concise Survey of Computer Methods, 397 p.
Studentlitteratur, Lund, Sweden, ISBN 91-44-07881-1, 1974.
\url{http://www.naur.com/Conc.Surv.html}}, ``Concise Survey of Computer Methods'', he
parts from the concept that \emph{data} is ``a representation of facts or ideas in a
formalised manner capable of being communicated or manipulated by some
process.''\footnote{I. H. Gould (ed.): ‘IFIP guide to concepts and terms in data
processing’, North-Holland Publ. Co., Amsterdam, 1971.} Note however that his view of the
science only ``deals with data [\dots] while the relation of data to what they represent
is delegated to other fields and sciences.''

\paragraph{William Cleveland (born 1943)}

In 2001, a prominent statistician used the term ``data science" in his work to describe a
new discipline that comes from his ``plan to enlarge the major areas of technical work of
the field of statistics\footnote{W. S. Cleveland. Data Science: An Action Plan for
Expanding the Technical Areas of the Field of Statistics. ISI Review, 69:21–26, 2001.}.''
In 2014, that work was republished\footnote{W. S. Cleveland.
Data Science: An Action Plan for the Field of Statistics. Statistical Analysis and Data
Mining, 7:414–417, 2014. reprinting of 2001 article in ISI Review, Vol 69.}.
He advocates the expansion of statistics beyond theory into technical areas, significantly
changing statistics.  Thus, it warranted a new name.

As a result, William Swain Cleveland II is credited to define data science as it is most
used today. He is a highly influential figure in the fields of statistics, machine
learning, data visualization, data analysis for multidisciplinary studies, and high
performance computing for deep data analysis.

\paragraph{Buzzword or a new science?}

Be aware that literature has no consensus on the definition of data science, and it is still considered
by some to be a buzzword\footnote{Press, Gil. "Data Science: What's The Half-Life of a
Buzzword?". Forbes. Available at
\url{https://www.forbes.com/sites/gilpress/2013/08/19/data-science-whats-the-half-life-of-a-buzzword/}}.

In \cref{chap:data}, I dare to provide a (yet another\dots) definition for the term.  I
argue that its object of study can be precisely established to support it as a new
science.

\section{Historical markers}

\section{Timeline}

\subsection{Timeline of data collection}

\begin{itemize}
  \item Pre-digital Age
  \item Digital Age
  \item Formal Age
  \item Integrated Age
  \item Ubiquitous Age
\end{itemize}

\subsection{Timeline of data analysis}
