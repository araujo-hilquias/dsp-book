\chapter{Data science project}

\chapterprecishere{Figured I could throw myself a pity party or go back to school and
learn the computers. \par\raggedleft--- \textup{Don Carlton}, Monsters University}

First of all, a data science project is a software project.  The difference between a data
science software and a traditional software is that some components of the former is
constructed from data.  This means that part of the solution cannot be designed from the
knowledge of the domain expert, but must be learned from the data.  (Alternatively, the
cost of designing the solution is too high, and it is more efficient to learn it from the data.)

One good example of a data science project is a spam filter.  The spam filter is a software
that classifies emails into two categories: spam and non-spam.  The software is trained
using a set of emails that are already classified as spam or non-spam.  The software is
then used to classify new emails.  The software is a data science software because the
classification algorithm is learned from the data, i.e. the filters are not designed ``by
hand''.

% TODO:
% - the need of methodologies for data science projects
% - CRISP-DM
% - Nina's approach
% - Agile
% - Problems and advantages of SCRUM
% - Our approach

\section{CRISP-DM}

CRISP-DM is a methodology for data science projects.  It is an acronym for Cross Industry
Standard Process for Data Mining.  It is a methodology that was developed in the 1990s, and
it is still widely used today.

\section{\citeauthor{Zumel2019}'s approach}

\textcite{Zumel2019} propose a methodology for data science projects.

They define five roles:
\begin{itemize}
  \item Project sponsor: represents the business interests and champions the project;
  \item Client: represents the end users’ interests and is the domain expert;
  \item Data scientist: sets and executes the analytic strategy and communicates with the
    sponsor and the client;
  \item Data architect: manages data and data storage and sometimes manages data
    collection;
  \item Operations: manages infrastructure and deploys final project results.
\end{itemize}

\citeauthor{Zumel2019}'s model is similar to CRISP-DM, but emphasizes that back-and-forth
is possible at any stage of the process.

\section{Agile}

Agile is a methodology for software development.  It is an alternative to the waterfall
methodology.  The waterfall methodology is a sequential design process, where each phase
must be completed before the next phase can begin.

The four values of agile manifesto are:
\begin{itemize}
  \item Individuals and interactions over processes and tools;
  \item Working software over comprehensive documentation;
  \item Customer collaboration over contract negotiation;
  \item Responding to change over following a plan.
\end{itemize}

\section{SCRUM}

SCRUM is an agile framework for software development.  It is a process framework for
managing complex projects.  It is a lightweight process framework, which means that it
provides just enough guidance to be effective.

Many consider that SCRUM is not adequate for data science projects.  The main reason is
that SCRUM is designed for projects where the requirements are known in advance.  Also,
that data science projects have exploratory phases, which are not well supported by SCRUM.

I argue that this view is wrong.  SCRUM is a framework, and it is designed to be adapted to
the needs of the project.  SCRUM is not a rigid process.  In the following, I propose an
extension to SCRUM that makes it suitable for data science projects.

(In real-world, most developers do not have hacking-level skills.  They are not autonomous
enough to work without a plan.  This is especially true for ``data scientists'', who are
often not even developers.  SCRUM is a good compromise between the need for autonomy and
the need for a detailed plan.  Project methodology is needed to ensure that the project is
completed in time and within budget.)

\section{Our approach}

We propose an extension to SCRUM that makes it suitable for data science projects.  The
extension is based on the following observations:
\begin{itemize}
  \item Data science projects have exploratory phases;
  \item \dots
\end{itemize}

Two other values must be added to the agile manifesto:
\begin{itemize}
  \item Model confidence/understanding over model performance;
  \item Beware of interactive environments.
\end{itemize}

\subsection{The roles}

Combine SCRUM roles with the roles defined by \textcite{Zumel2019}.

\subsection{The principles of our approach}

\begin{enumerate}
  \item Modularize the solution. Usually, in four main modules: frontend, backend,
    dataset, and model search.  The frontend is the user interface.  The backend is the
    server-side code.  The dataset is the data that is used to train the model.  The model
    search is the code that searches for the best model.
  \item Version control everything.  This includes the code, the data, and the
    documentation. The most used tool for code version control is Git.  For datasets,
    extensions to Git exist, such as DVC\footnote{\url{https://dvc.org/}}.  One important aspect
    is to version control the model search code.  Interactive environments such as Jupyter
    notebooks are not suitable for this purpose.  They can be used to draft the code, but
    the final version must be version controlled.
  \item Continuous integration and continuous deployment.  This means that the code is
    automatically (or at least semi-automatically) tested and deployed.  The backend and
    frontend code is tested using unit tests.  The model search code is tested using
    validation methods such as cross-validation and Bayesian analysis on the discovered
    models.  Usually the model search code is very computationally intensive, and it is
    not feasible to run it on every commit.  Instead, it is run periodically, for example
    once a day.  If the clould infrastructure required to run the model search code is not
    available to automate validation and deploymen, at least make sure that the code is
    easily runnable.  This means that the code must be well documented, and that the
    required infrastructure must be well documented.  Also aggregate commands using a
    Makefile or a similar tool.  Pay attention on the dependences between dataset and the
    model training.  If the dataset changes significantly, not only the deployed model
    must be retrained, but the model search algorithm may need to be rethought.
  \item Reports as deliverables.  During sprints, the deliverables of data exploration are
    reports.
  \item Setup quantitative goals.  Do not fall on the trap of forever improving the model.
    Instead, setup quantitative goals for the model performance.  For example, the model
    must have a precision of at least 90\%.  Once you reach the goal, prioritize other
    tasks.
  \item Measure \emph{exactly} what you want.  During model validation, use your own
    metrics based on the project goals.  Usually, more than one metric is needed, and they
    might be conflicting.  Use strategies to balance the metrics, such as Pareto
    optimization.  Beware of the metrics that are most used in the literature.  They might not
    be suitable for your project.  Notice that during model training, some methods are
    limited to the loss functions that they can optimize.  If possible, choose a method
    that can optimize the loss function that you want.  Even if you are not explicitly
    optimizing the wanted metric, you might find a model that performs well on that metric.
    That is a reason validation is important.
  \item Report model stability and performance variance.  Understanding the limitations
    and characteristics of the model is more important than the model performance.  For
    example, if the model performance is high, but the model is unstable, it is not
    suitable for production.  Also, in some scenarios, interpretability is more important than
    performance.
  \item In user interface, mask data-science-specific terminology.  Usually, data science
    software gives the user the option to choose the model.  In order to avoid confusion,
    the user interface must mask the data-science-specific terminology.  This helps non
    experts to use the software consciously.
  \item Monitor model performance in production.  If possible setup feedback from the user
    interface.  Avoid automation of model releases because concept drift usually requires
    exploratory analysis.
  \item Use the appropriate backend.  REST API vs websocket.
\end{enumerate}
