\chapter*{About this book}

\begin{parwithqr}{https://github.com/verri/dsp-book}
  I intend to make this book forever free and open-source. You can find (and contribute
  to) the source code at \href{\aurl}{github.com/verri/dsp-book}.
\end{parwithqr}

\vfill

\begin{parwithqr}{https://www.buymeacoffee.com/verri}
  If you like this book, consider \emph{buying me a coffee} at
  \href{\aurl}{buymeacoffee.com/verri}.
\end{parwithqr}

\vfill

\begin{parwithqr}{https://github.com/verri/dsp-book/issues}
  If you find any typos, grammar errors, incomplete material or have any suggestions,
  please open an issue at \href{\aurl}{github.com/verri/dsp-book/issues}.
\end{parwithqr}

\vfill

\begin{parwithqr}{https://comp.ita.br/\~verri/ds-book-print}
  Students can found a printable version (A4 paper, double-sided, short-edge spiral
  binding) of this book at \href{\aurl}{comp.ita.br/\textasciitilde{}verri/ds-book-print}.
\end{parwithqr}

\newpage
This book comprises the lectures notes of the course PO-235 Data Science Project.
I hope someday it becomes an actual book. For now, beware many typos, grammar errors, ugly
typesetting, disconnected material, etc.

Also, it is important to highlight that:
\begin{itemize}
  \item This is not a Machine Learning book, and I do not intend to explain how specific
    ML algorithms work.
  \item This contains some kind of introductory material on data science.  Although I will
    introduce the fundamental concepts, I expect you have strong mathematical and
    statistics background.
  \item An artificial constraint I have imposed in the material (for the sake of the
    course) is that I will only consider \emph{predictive methods}, more specifically
    inductive ones. I will not address topics such as clustering, association-rules
    mining, transductive learning, anomaly detection, time series forecasting, reinforced
    learning, etc.
\end{itemize}

I have decided to work on this material because the books I like on data science are
either
\begin{itemize}
  \item too broad and too shallow, in the sense they hide many mathematical foundations
    and focus on just explaining what data science is and where it is applied;
  \item too tool-centric, in the sense that they focus only on a specific toolbox or
    programming language; or
  \item too machine-learning-y, exposing many machine learning algorithms and missing the
    foundations of learning.
\end{itemize}

So\dots, I expect my approach on the subject provide:
\begin{itemize}
  \item awareness of all steps in a data science project;
  \item deeper focus (than most books) on data transformation, describing the semantics of dataset
    operators instead of restraining ourselves with a specific tool;
  \item deeper focus (than most books) on why machine learning works, increasing awareness of its pitfalls and
    limitations;
  \item deeper focus (than most books) on correct evaluation and validation
    (pre-deployment) of machine learning models.
\end{itemize}

This book will cover the following material:
\begin{itemize}
  \item Brief history of data science.
  \item Background topics.
  \item Fundamental data concepts.
  \item Stages in a data science project.
  \item Data Infrastructure.
  \item Data integration from multiple sources.
  \item Data engineering and shaping.
  \item Inductive learning and principles of statistical learning theory.
  \item Application of Machine Learning models in real-world problems.
  \item Experimental planning for data science.
  \item Model evaluation and Bayesian analysis.
  \item Documentation and deployment.
  \item Ethical and legal issues in data science.
  \item Privacy-preserving computational approaches.
\end{itemize}
