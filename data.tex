\chapter{Fundamental data concepts}
\label{chap:data}

\chapterprecishere{The simple believes everything,
  \par\raggedleft but the prudent gives thought to his steps.
  \par\raggedleft--- \textup{Proverbs 14:15} (ESV)}

\begin{itemize}
  \item Variables, types etc
  \item Records
  \item Tidy data
\end{itemize}

A useful start point for someone studying data science is the definition of the term
itself.

For \textcite{Zumel2019}, \emph{``data science is a cross-disciplinary practice that draws
on methods from data engineering, descriptive statistics, data mining, machine learning,
and predictive analytics.''}  They compare the area with the operations research, stating
that data science focuses on implementing data-driven decisions and managing their
consequences.

\textcite{Hickham2023} state that \emph{``data science is an exciting discipline that
allows you to transform raw data into understanding, insight, and knowledge.''}

I find the first definition too restrictive once new methods and techniques are always
under development.  We never know when new ``data-related'' methods will become obsolete
or a trend.  Also, \textcite{Zumel2019}'s view gives the impression that data science is a
operations research subfield.  Although I will not try to prove otherwise, I think it will
be much more useful to see it as an independent field of study.  Obviously, there will be
many intersections between both areas (and many other areas as well.)  Because of such
intersections, I will try my best to keep definitions and
terms standardized throughout chapters, sometimes avoiding popular terms that may generate
ambiguities or confusion.

The second one is not really a definition.  However, it states clearly \emph{what} data
science enables us to do.  From these thoughts, let's define the term.

\begin{displayquote}
  \em
  Data science is the study of computational methods to extract knowledge from
  measurable phenomena.
\end{displayquote}

I want to highlight the meaning of some terms in this definition.  \emph{Computational methods} means
that data science methods use computers to handle data and perform the calculations.
\emph{Knowledge} means information that humans can easily understand or apply to solve
problems.  \emph{Measurable phenomena} are events or processes where raw data can be
quantified in some way.  \emph{Raw data} are data collected directly from some source and
that have not been subject to any other manipulation by a software program or a human
expert.

\textcite{Kelleher2018} summarize very well the challenges data science takes up:
``extracting non-obvious and useful patterns from large data sets [\dots]; capturing,
cleaning, and transforming [\dots] data; [storing and processing] big [\dots] data sets;
and questions related to data ethics and regulation.''

Data science contrasts with conventional sciences.  Usually, a ``science'' is named after
its object of study.  Biology is the study of the life, Earth science studies the planet
Earth, and so on.  I argue that data science does not study data itself, but how we can
``listen'' them to understand a phenomenon.  The conventional scientific paradigm is
essentially model-driven: we observe a phenomenon related to the object of study, we
reason the possible explanation (the model or hypothesis,) and we validate our hypothesis
(most of the time using data, though.)  In data science, however, we extract the knowledge
directly and primarily from the data.  The expert knowledge and reasoning may be taken
into account, but we give data the opportunity to surprise us.  Thus, the objects of the
study in data science are the computational methods and processes that can extract
reliable and ethical knowledge from huge amounts of data.

\def\verrids{(0,0) circle (20mm)}
\def\verrist{(-2.5,0) circle (15mm)}
\def\verride {(2.5,0) circle (15mm)}
\def\verrics {(0,-2.5) circle (15mm)}

\begin{figure}
  \centering
  \begin{tikzpicture}
    \begin{scope}
      \clip \verrids;
      \fill[filled] \verrist;
      \fill[filled] \verride;
      \fill[filled] \verrics;
    \end{scope}
    \draw[outline] \verrids node(ds) {};
    \draw[outline] \verrist node {statistics};
    \draw[outline] \verride node {domain expertise};
    \draw[outline] \verrics node {computer science};
    \node[anchor=north,above] at (0,2) {data science};
  \end{tikzpicture}
  \caption{
    My view of data science.  Data science is an entire new science.  Being a new science
    does not mean that its basis is built from the ground up.  Most of the subjects in
    data science come from other sciences, but its object of study (computational methods
    to extract knowledge from measurable phenomena) is particular enough to unfold
    new scientific questions -- such as data ethics, data collection, etc.
  }
\end{figure}
