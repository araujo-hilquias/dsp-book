\chapter{Experimental planning}
\label{chap:planning}

\chapterprecishere{%
  All models are wrong, but some are useful.
  \par\raggedleft--- \textup{George E. P. Box}, Robustness in Statistics}

Once we have defined what an inductive problem is, we can start to think about how to
solve it in practical terms.

In this chapter, we present the experimental planning one can use in the data-driven
parts of a data science project.  \emph{Experimental planing}  in the context of data
science involves designing and organizing experiments to gather performance data
systematically in order to reach specific goals or test hypotheses.

The reason we need to plan experiments is that data science is experimental, i.e. we
usually lack a theoretical model that can predict the outcome of a given algorithm on a
given dataset.  This is why we need to run experiments to gather data and make
inferences from it.

There is not a single way to plan experiments, but there are some common steps that can
be followed to design a good experimental plan.  In this chapter, we present a
framework for experimental planning that can be used in most data science projects.

\section{Elements of an experimental plan}

There are important elements that should be considered when designing an experimental
plan.  These elements are:
\begin{itemize}
  \item \textbf{Hypothesis}: The main question that the experiment aims to validate.
  \item \textbf{Data}: The dataset that will be used in the experiment.
  \item \textbf{Strategies}: The algorithms or methods that will be used to find the
    model\footnote{Remember, in an inductive learning problem, we look for a model to
    map inputs to outputs the best we can.} that solves the problem.
  \item \textbf{Performance measure}: The metric that will be used to evaluate the
    performance of the model.
\end{itemize}

A general example of a description of an experimental plan is ``What is the probability of
the strategy $A$ to find a model that reaches a performance $X$ in terms of metric $Y$ in
the real-world given dataset $Z$ as training set (assuming $Z$ is a representative
dataset)?''

Another example is ``Is strategy $A$ better than strategy $B$ for finding a model that
predicts the output of dataset $C$ given $D$ as a training set in terms of metric $E$?''

We consider these two cases: \emph{estimating expected performance} and \emph{comparing
strategies}.

% vim: spell spelllang=en
