\chapter*{Course plan}

For administrative reasons, I present the course plan for PO-235 and CMC-1X.

\newpage
\newgeometry{margin=1cm}
\thispagestyle{empty}
\section*{PO-235 Data Science Project}

\emph{Course plan}

\nth{1} Semester of \the\year{}

Prof. Filipe A. N. Verri

\paragraph{Number of students:} Approx. 20 (hopefully close to 10)

\paragraph{Course load:} 3--0--0--4

\paragraph{Requirements:} Advanced programming skills, strong statistics background, and
beginner level machine learning skills.

\paragraph{Course program:}
Brief history of Data Science.  Fundamental data concepts. Stages in a Data Science
project.  Data Infrastructure. Data integration from multiple sources. Data engineering
and shaping.  Inductive learning and principles of Statistical Learning Theory.
Application of Machine Learning models in real-world problems.  Experimental planning for
Data Science. Model evaluation and Bayesian analysis.  Documentation and deployment.
Ethical and legal issues in Data Science.  Privacy-preserving computational approaches.

\paragraph{Goals:}
Providing the theoretical background and the practical concepts to develop an end-to-end
data science project for an inductive task.

\paragraph{Teaching methodology:}
Expository classes in common classroom, using whiteboard, slide presentations, coding
examples, books and scientific papers. Supplementary didactic materials will be available
in Google Classroom. The development of the case study will happen during home study
hours, including programming and scientific paper writing.

\paragraph{Grading:} Two written tests in the \nth{1} and one in the \nth{2} quarter.
Scientific paper and oral presentation about the case study for the final exam.

\paragraph{Bibliography:}
\begin{itemize}
  \item Nina Zumel and John Mount. Practical Data Science with R. Manning, 2nd Edition, 2019.
  \item Hadley Wickham and Garret Grolemund, R for Data Science: Import, Tidy, Transform, Visualize, and Model Data. O’Reilly Media, 2017.
  \item John D. Kelleher and Brendan Tierney. Data Science, MIT Press, 2018.
\end{itemize}

\thispagestyle{empty}
\paragraph{Must read:}
\begin{itemize}
  \item In-progress textbook at \url{https://comp.ita.br/~verri/dsp-book}.
  \item \fullcite{Vapnik1999}.
  \item \fullcite{Benavoli2017}.
\end{itemize}

\paragraph{Calendar:} The expected schedule is presented below.

\begin{center}
  \begin{tabular}{ll}
    \toprule
    \multicolumn{2}{c}{\bf \nth{1} Quarter} \\
    \midrule
    Week & Topics \\
    \midrule
    1 & Brief history of Data Science and background topics \\
    \midrule
    2 & \bf Written test \\
    \midrule
    3 & Fundamental data concepts and stages in a Data Science project \\
    \midrule
    4 & \multirow{2}{*}{Inductive learning and Statistical Learning Theory} \\
    5 &  \\
    \midrule
    6 & Data infrastructure and data integration from multiple sources \\
    \midrule
    7 & Data engineering and shaping \\
    \midrule
    8 & \bf Written test \\
    \bottomrule
  \end{tabular}
\end{center}

\begin{center}
  \begin{tabular}{ll}
    \toprule
    \multicolumn{2}{c}{\bf \nth{2} Quarter} \\
    \midrule
    Week & Topics \\
    \midrule
    1 & \multirow{2}{*}{Application of Machine Learning models in real-world problems} \\
    2 &  \\
    \midrule
    3 & \multirow{2}{*}{Experimental planning for Data Science} \\
    4 & \\
    \midrule
    5 & \multirow{2}{*}{Model evaluation and Bayesian analysis} \\
    6 & \\
    \midrule
    7 & \bf Written test \\
    \midrule
    \multirow{3}{*}{8} & Documentation and deployment \\
      & Ethical and legal issues in Data Science \\
      & Privacy-preserving computational approaches \\
    \bottomrule
  \end{tabular}
\end{center}

Case studies will be presented during exam weeks.

\newpage
\newgeometry{margin=1cm}
\thispagestyle{empty}
\section*{CMC-1X Practical Data Science}

\emph{Course plan}

\nth{1} Semester of \the\year{}

Prof. Filipe A. N. Verri

\paragraph{Number of students:} Approx. 20

\paragraph{Course load:} 2--0--1--5

\paragraph{Requirements:} CMC-13 or CMC-15

\paragraph{Course program:}
Brief history of Data Science. Stages in a Data Science project. Tidy Data. Data
integration from multiple sources. Data engineering and shaping. Inductive learning and
Statistical Learning Theory. Experimental planning for Data Science. Model evaluation and
Bayesian Analysis. Documentation and deployment. Privacy-preserving computational
approaches.

\paragraph{Goals:}
Further studying the practical aspects of Data Science (in relation to CMC-13) and providing
the mathematical foundations to ensure the correct usage of Data Science techniques.

The specific goals are:
\begin{itemize}
  \item Understanding the steps in a Data Science project;
  \item Developing an end-to-end case study, including data collection, data transformation,
    inductive learning, validation, documentation, and deployment; and
  \item Critically evaluate the results and implications of the case study.
\end{itemize}

\paragraph{Teaching methodology:}
Expository classes in common classroom, using whiteboard, slide presentations, coding
examples, books and scientific papers. Supplementary didactic materials will be available
in Google Classroom. The development of the case study will happen during laboratory
classes and home study hours, including programming and writing essays.

\paragraph{Grading:} One written test in the \nth{1} and one in the \nth{2} quarter.
Essay describing the development of the case study and oral presentation for the final
exam.

\paragraph{Bibliography:}
\begin{itemize}
  \item Nina Zumel and John Mount. Practical Data Science with R. Manning, 2nd Edition, 2019.
  \item Hadley Wickham and Garret Grolemund, R for Data Science: Import, Tidy, Transform, Visualize, and Model Data. O’Reilly Media, 2017.
  \item John D. Kelleher and Brendan Tierney. Data Science, MIT Press, 2018.
\end{itemize}

\thispagestyle{empty}
\paragraph{Recommended reading:}
\begin{itemize}
  \item In-progress textbook at \url{https://comp.ita.br/~verri/dsp-book}.
  \item \fullcite{Vapnik1999}.
  \item \fullcite{Benavoli2017}.
\end{itemize}

\paragraph{Calendar:} The expected schedule is presented below.

\begin{center}
  \begin{tabular}{ll}
    \toprule
    \multicolumn{2}{c}{\bf \nth{1} Quarter} \\
    \midrule
    Week & Topics \\
    \midrule
    1 & Brief history of Data Science and CMC-13 review \\
    2 & Stages in a Data Science project \\
    3 & Tidy Data and data integration from multiple sources \\
    4 & Data engineering and shaping \\
    5 & \multirow{2}{*}{Inductive learning and Statistical Learning Theory} \\
    6 &  \\
    7 & Case study discussion and definitions \\
    8 & \bf Written test \\
    \bottomrule
  \end{tabular}
\end{center}

\begin{center}
  \begin{tabular}{ll}
    \toprule
    \multicolumn{2}{c}{\bf \nth{2} Quarter} \\
    \midrule
    Week & Topics \\
    \midrule
    1 & Experimental planning for Data Science \\
    2 & Model evaluation \\
    3 & Bayesian Analysis \\
    4 & Documentation and deployment \\
    5 & Privacy-preserving computational approaches \\
    6 & \bf Written test \\
    7 & \multirow{2}{*}{\bf Presentations and discussions} \\
    8 & \\
    \bottomrule
  \end{tabular}
\end{center}

\restoregeometry
